\documentclass[12pt]{extarticle}
\usepackage[utf8]{inputenc}
\usepackage[T1]{fontenc}
\usepackage{graphicx}
\usepackage{geometry}
\usepackage{fancyhdr}
\usepackage{lipsum}
\usepackage[table]{xcolor}
\usepackage{booktabs}
\usepackage{amsmath}
\usepackage{enumitem}
\usepackage{afterpage}

% Définir les marges
\geometry{
left=2cm,
right=2cm,
top=2cm,
bottom=2cm
}

% Pied de page
\fancypagestyle{plain}{
\fancyhf{}
\renewcommand{\headrulewidth}{0pt}
\fancyfoot[C]{\thepage}
}

\pagestyle{fancy}
\fancyhf{}
\fancyfoot[C]{\thepage}
\fancyhead[R]{Document privé et confidentiel.}

\begin{document}

% Page de garde
\begin{titlepage}
    \centering
    \vspace*{3cm}
    \vspace{1cm}
    {\Huge\bfseries Convention d'audit \par}
    \vspace{2cm}
    {\Large \bfseries AUDITORCOMPANY \par}
    \vfill
    \vspace{2cm}
\end{titlepage}

\newpage
\section*{Présentation}
La convention d’audit a pour rôle de protéger l’audité, le commanditaire et l’auditeur. Elle s’appuie sur les lois françaises et sur le référentiel d’exigence de l’ANSSI. Sans la signature de toutes les parties de cette convention, l’audit ne pourra pas débuter.

\newpage
\section*{Périmètre}
L’audit va se concentrer sur les cibles suivantes :
\begin{itemize}
    \item cible.com
    \item 192.168.1.1
\end{itemize}

Ces cibles sont hébergées par {{HEBERGEUR}}. Les serveurs Web sont dans des environnements de préproduction.

\newpage
\section*{Modalité}
L’audit se déroulera du \textbf{ma\_date\_deb} jusqu’au \textbf{ma\_date\_fin}. Les tests d’intrusion se déroulent entre \textit{8h00 et 18h00} heure de Paris, uniquement les week-ends. Les ressources à fournir incluent:
\begin{itemize}
    \item Accès VPN ou Internet ;
    \item Ressources sur les mécanismes de la cible ;
    \item Comptes utilisateur avec différents droits ;
    \item Contacts de sécurité et technique.
\end{itemize}

Le rapport sera livré le \textit{31 décembre 2022}, et la réunion de restitution entre le \textit{01 janvier 2021} et \textit{30 décembre 2023}.

\newpage
\section*{Diffusion}
\begin{tabular}{|l|l|l|l|}
\hline
\textbf{Nom Prénom} & \textbf{Rôle} & \textbf{Droits} & \textbf{Contact} \\ \hline
Perez David & Auditeur & Auditeur & admin@dacodhack.com \\ \hline
John Smith & RSSI & Accès aux vulnérabilités & contact@mail.com \\ \hline
\end{tabular}

\newpage
\section*{Actions et données collectées}
L’auditeur s’appuiera sur la méthodologie OWASP WTSG3. Le commanditaire exclut certains tests (suppression de données, latéralisation, etc.). Les données collectées seront supprimées après restitution.

\newpage
\section*{SIGNATURE DE LA NOM\_DOCUMENT}
Toute modification ou annulation du périmètre doit être validée par les deux parties.

\vfill
\textbf{Pour le Client :}\par
Moi, \textit{clientrepresentative}, signe en tant que \textit{CLIENT\_TITLE}.\par
\vspace{3cm}
Signature :\par
\textbf{Pour l'Auditeur :}\par
Moi, David Perez, signe en tant que \textit{CEO}.\par
\vspace{3cm}
Signature :

\end{document}
